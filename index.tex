% Options for packages loaded elsewhere
\PassOptionsToPackage{unicode}{hyperref}
\PassOptionsToPackage{hyphens}{url}
\PassOptionsToPackage{dvipsnames,svgnames,x11names}{xcolor}
%
\documentclass[
  letterpaper,
  DIV=11,
  numbers=noendperiod]{scrreprt}

\usepackage{amsmath,amssymb}
\usepackage{lmodern}
\usepackage{iftex}
\ifPDFTeX
  \usepackage[T1]{fontenc}
  \usepackage[utf8]{inputenc}
  \usepackage{textcomp} % provide euro and other symbols
\else % if luatex or xetex
  \usepackage{unicode-math}
  \defaultfontfeatures{Scale=MatchLowercase}
  \defaultfontfeatures[\rmfamily]{Ligatures=TeX,Scale=1}
\fi
% Use upquote if available, for straight quotes in verbatim environments
\IfFileExists{upquote.sty}{\usepackage{upquote}}{}
\IfFileExists{microtype.sty}{% use microtype if available
  \usepackage[]{microtype}
  \UseMicrotypeSet[protrusion]{basicmath} % disable protrusion for tt fonts
}{}
\makeatletter
\@ifundefined{KOMAClassName}{% if non-KOMA class
  \IfFileExists{parskip.sty}{%
    \usepackage{parskip}
  }{% else
    \setlength{\parindent}{0pt}
    \setlength{\parskip}{6pt plus 2pt minus 1pt}}
}{% if KOMA class
  \KOMAoptions{parskip=half}}
\makeatother
\usepackage{xcolor}
\setlength{\emergencystretch}{3em} % prevent overfull lines
\setcounter{secnumdepth}{5}
% Make \paragraph and \subparagraph free-standing
\ifx\paragraph\undefined\else
  \let\oldparagraph\paragraph
  \renewcommand{\paragraph}[1]{\oldparagraph{#1}\mbox{}}
\fi
\ifx\subparagraph\undefined\else
  \let\oldsubparagraph\subparagraph
  \renewcommand{\subparagraph}[1]{\oldsubparagraph{#1}\mbox{}}
\fi

\usepackage{color}
\usepackage{fancyvrb}
\newcommand{\VerbBar}{|}
\newcommand{\VERB}{\Verb[commandchars=\\\{\}]}
\DefineVerbatimEnvironment{Highlighting}{Verbatim}{commandchars=\\\{\}}
% Add ',fontsize=\small' for more characters per line
\usepackage{framed}
\definecolor{shadecolor}{RGB}{241,243,245}
\newenvironment{Shaded}{\begin{snugshade}}{\end{snugshade}}
\newcommand{\AlertTok}[1]{\textcolor[rgb]{0.68,0.00,0.00}{#1}}
\newcommand{\AnnotationTok}[1]{\textcolor[rgb]{0.37,0.37,0.37}{#1}}
\newcommand{\AttributeTok}[1]{\textcolor[rgb]{0.40,0.45,0.13}{#1}}
\newcommand{\BaseNTok}[1]{\textcolor[rgb]{0.68,0.00,0.00}{#1}}
\newcommand{\BuiltInTok}[1]{\textcolor[rgb]{0.00,0.23,0.31}{#1}}
\newcommand{\CharTok}[1]{\textcolor[rgb]{0.13,0.47,0.30}{#1}}
\newcommand{\CommentTok}[1]{\textcolor[rgb]{0.37,0.37,0.37}{#1}}
\newcommand{\CommentVarTok}[1]{\textcolor[rgb]{0.37,0.37,0.37}{\textit{#1}}}
\newcommand{\ConstantTok}[1]{\textcolor[rgb]{0.56,0.35,0.01}{#1}}
\newcommand{\ControlFlowTok}[1]{\textcolor[rgb]{0.00,0.23,0.31}{#1}}
\newcommand{\DataTypeTok}[1]{\textcolor[rgb]{0.68,0.00,0.00}{#1}}
\newcommand{\DecValTok}[1]{\textcolor[rgb]{0.68,0.00,0.00}{#1}}
\newcommand{\DocumentationTok}[1]{\textcolor[rgb]{0.37,0.37,0.37}{\textit{#1}}}
\newcommand{\ErrorTok}[1]{\textcolor[rgb]{0.68,0.00,0.00}{#1}}
\newcommand{\ExtensionTok}[1]{\textcolor[rgb]{0.00,0.23,0.31}{#1}}
\newcommand{\FloatTok}[1]{\textcolor[rgb]{0.68,0.00,0.00}{#1}}
\newcommand{\FunctionTok}[1]{\textcolor[rgb]{0.28,0.35,0.67}{#1}}
\newcommand{\ImportTok}[1]{\textcolor[rgb]{0.00,0.46,0.62}{#1}}
\newcommand{\InformationTok}[1]{\textcolor[rgb]{0.37,0.37,0.37}{#1}}
\newcommand{\KeywordTok}[1]{\textcolor[rgb]{0.00,0.23,0.31}{#1}}
\newcommand{\NormalTok}[1]{\textcolor[rgb]{0.00,0.23,0.31}{#1}}
\newcommand{\OperatorTok}[1]{\textcolor[rgb]{0.37,0.37,0.37}{#1}}
\newcommand{\OtherTok}[1]{\textcolor[rgb]{0.00,0.23,0.31}{#1}}
\newcommand{\PreprocessorTok}[1]{\textcolor[rgb]{0.68,0.00,0.00}{#1}}
\newcommand{\RegionMarkerTok}[1]{\textcolor[rgb]{0.00,0.23,0.31}{#1}}
\newcommand{\SpecialCharTok}[1]{\textcolor[rgb]{0.37,0.37,0.37}{#1}}
\newcommand{\SpecialStringTok}[1]{\textcolor[rgb]{0.13,0.47,0.30}{#1}}
\newcommand{\StringTok}[1]{\textcolor[rgb]{0.13,0.47,0.30}{#1}}
\newcommand{\VariableTok}[1]{\textcolor[rgb]{0.07,0.07,0.07}{#1}}
\newcommand{\VerbatimStringTok}[1]{\textcolor[rgb]{0.13,0.47,0.30}{#1}}
\newcommand{\WarningTok}[1]{\textcolor[rgb]{0.37,0.37,0.37}{\textit{#1}}}

\providecommand{\tightlist}{%
  \setlength{\itemsep}{0pt}\setlength{\parskip}{0pt}}\usepackage{longtable,booktabs,array}
\usepackage{calc} % for calculating minipage widths
% Correct order of tables after \paragraph or \subparagraph
\usepackage{etoolbox}
\makeatletter
\patchcmd\longtable{\par}{\if@noskipsec\mbox{}\fi\par}{}{}
\makeatother
% Allow footnotes in longtable head/foot
\IfFileExists{footnotehyper.sty}{\usepackage{footnotehyper}}{\usepackage{footnote}}
\makesavenoteenv{longtable}
\usepackage{graphicx}
\makeatletter
\def\maxwidth{\ifdim\Gin@nat@width>\linewidth\linewidth\else\Gin@nat@width\fi}
\def\maxheight{\ifdim\Gin@nat@height>\textheight\textheight\else\Gin@nat@height\fi}
\makeatother
% Scale images if necessary, so that they will not overflow the page
% margins by default, and it is still possible to overwrite the defaults
% using explicit options in \includegraphics[width, height, ...]{}
\setkeys{Gin}{width=\maxwidth,height=\maxheight,keepaspectratio}
% Set default figure placement to htbp
\makeatletter
\def\fps@figure{htbp}
\makeatother
\newlength{\cslhangindent}
\setlength{\cslhangindent}{1.5em}
\newlength{\csllabelwidth}
\setlength{\csllabelwidth}{3em}
\newlength{\cslentryspacingunit} % times entry-spacing
\setlength{\cslentryspacingunit}{\parskip}
\newenvironment{CSLReferences}[2] % #1 hanging-ident, #2 entry spacing
 {% don't indent paragraphs
  \setlength{\parindent}{0pt}
  % turn on hanging indent if param 1 is 1
  \ifodd #1
  \let\oldpar\par
  \def\par{\hangindent=\cslhangindent\oldpar}
  \fi
  % set entry spacing
  \setlength{\parskip}{#2\cslentryspacingunit}
 }%
 {}
\usepackage{calc}
\newcommand{\CSLBlock}[1]{#1\hfill\break}
\newcommand{\CSLLeftMargin}[1]{\parbox[t]{\csllabelwidth}{#1}}
\newcommand{\CSLRightInline}[1]{\parbox[t]{\linewidth - \csllabelwidth}{#1}\break}
\newcommand{\CSLIndent}[1]{\hspace{\cslhangindent}#1}

\KOMAoption{captions}{tableheading}
\makeatletter
\makeatother
\makeatletter
\@ifpackageloaded{bookmark}{}{\usepackage{bookmark}}
\makeatother
\makeatletter
\@ifpackageloaded{caption}{}{\usepackage{caption}}
\AtBeginDocument{%
\ifdefined\contentsname
  \renewcommand*\contentsname{Table of contents}
\else
  \newcommand\contentsname{Table of contents}
\fi
\ifdefined\listfigurename
  \renewcommand*\listfigurename{List of Figures}
\else
  \newcommand\listfigurename{List of Figures}
\fi
\ifdefined\listtablename
  \renewcommand*\listtablename{List of Tables}
\else
  \newcommand\listtablename{List of Tables}
\fi
\ifdefined\figurename
  \renewcommand*\figurename{Figure}
\else
  \newcommand\figurename{Figure}
\fi
\ifdefined\tablename
  \renewcommand*\tablename{Table}
\else
  \newcommand\tablename{Table}
\fi
}
\@ifpackageloaded{float}{}{\usepackage{float}}
\floatstyle{ruled}
\@ifundefined{c@chapter}{\newfloat{codelisting}{h}{lop}}{\newfloat{codelisting}{h}{lop}[chapter]}
\floatname{codelisting}{Listing}
\newcommand*\listoflistings{\listof{codelisting}{List of Listings}}
\makeatother
\makeatletter
\@ifpackageloaded{caption}{}{\usepackage{caption}}
\@ifpackageloaded{subcaption}{}{\usepackage{subcaption}}
\makeatother
\makeatletter
\@ifpackageloaded{tcolorbox}{}{\usepackage[many]{tcolorbox}}
\makeatother
\makeatletter
\@ifundefined{shadecolor}{\definecolor{shadecolor}{rgb}{.97, .97, .97}}
\makeatother
\makeatletter
\makeatother
\ifLuaTeX
  \usepackage{selnolig}  % disable illegal ligatures
\fi
\IfFileExists{bookmark.sty}{\usepackage{bookmark}}{\usepackage{hyperref}}
\IfFileExists{xurl.sty}{\usepackage{xurl}}{} % add URL line breaks if available
\urlstyle{same} % disable monospaced font for URLs
\hypersetup{
  pdftitle={jump-start-book},
  pdfauthor={Steve Simon},
  colorlinks=true,
  linkcolor={blue},
  filecolor={Maroon},
  citecolor={Blue},
  urlcolor={Blue},
  pdfcreator={LaTeX via pandoc}}

\title{jump-start-book}
\author{Steve Simon}
\date{10/3/23}

\begin{document}
\maketitle
\ifdefined\Shaded\renewenvironment{Shaded}{\begin{tcolorbox}[interior hidden, breakable, frame hidden, enhanced, borderline west={3pt}{0pt}{shadecolor}, boxrule=0pt, sharp corners]}{\end{tcolorbox}}\fi

\renewcommand*\contentsname{Table of contents}
{
\hypersetup{linkcolor=}
\setcounter{tocdepth}{2}
\tableofcontents
}
\bookmarksetup{startatroot}

\hypertarget{preface}{%
\chapter*{Preface}\label{preface}}
\addcontentsline{toc}{chapter}{Preface}

\markboth{Preface}{Preface}

This book was a long time coming.

\bookmarksetup{startatroot}

\hypertarget{introduction}{%
\chapter{Introduction}\label{introduction}}

This is a great book.

\bookmarksetup{startatroot}

\hypertarget{picking-a-research-topic}{%
\chapter{Picking a research topic}\label{picking-a-research-topic}}

Insert text.

\hypertarget{step-1.-describe-what-bothers-you}{%
\section{Step 1. Describe what bothers
you}\label{step-1.-describe-what-bothers-you}}

Insert text.

\hypertarget{step-2.-read-what-others-have-said}{%
\section{Step 2. Read what others have
said}\label{step-2.-read-what-others-have-said}}

Insert text.

\hypertarget{step-3.-talk-about-your-idea-with-others}{%
\section{Step 3. Talk about your idea with
others}\label{step-3.-talk-about-your-idea-with-others}}

Insert text.

\hypertarget{the-fly-in-the-ointment}{%
\section{The fly in the ointment}\label{the-fly-in-the-ointment}}

Insert text.

\bookmarksetup{startatroot}

\hypertarget{selecting-your-research-hypothesis}{%
\chapter{Selecting your research
hypothesis}\label{selecting-your-research-hypothesis}}

Insert text.

\hypertarget{step-1.-choose-an-outcome-measure}{%
\section{Step 1. Choose an outcome
measure}\label{step-1.-choose-an-outcome-measure}}

Insert text.

\hypertarget{step-2.-define-your-patient-population}{%
\section{Step 2. Define your patient
population}\label{step-2.-define-your-patient-population}}

Insert text.

\hypertarget{step-3.-select-your-control-group}{%
\section{Step 3. Select your control
group}\label{step-3.-select-your-control-group}}

Insert text.

\hypertarget{the-fly-in-the-ointment-1}{%
\section{The fly in the ointment}\label{the-fly-in-the-ointment-1}}

Insert text.

\bookmarksetup{startatroot}

\hypertarget{selecting-your-research-design}{%
\chapter{Selecting your research
design}\label{selecting-your-research-design}}

Insert text.

\hypertarget{step-1.-decide-if-you-canshould-randomize}{%
\section{Step 1. Decide if you can/should
randomize}\label{step-1.-decide-if-you-canshould-randomize}}

Insert text.

\hypertarget{step-2.-consider-the-advantagesdisadvantages-of-retrospective-versus-prospective-research}{%
\section{Step 2. Consider the advantages/disadvantages of retrospective
versus prospective
research}\label{step-2.-consider-the-advantagesdisadvantages-of-retrospective-versus-prospective-research}}

Insert text.

\hypertarget{step-3.-think-about-the-most-efficient-way-to-select-subjects}{%
\section{Step 3. Think about the most efficient way to select
subjects}\label{step-3.-think-about-the-most-efficient-way-to-select-subjects}}

Insert text.

\hypertarget{the-fly-in-the-ointment-2}{%
\section{The fly in the ointment}\label{the-fly-in-the-ointment-2}}

Insert text.

\bookmarksetup{startatroot}

\hypertarget{selecting-your-sample-size}{%
\chapter{Selecting your sample size}\label{selecting-your-sample-size}}

This is a book created from markdown and executable code.

See Knuth (1984) for additional discussion of literate programming.

\begin{Shaded}
\begin{Highlighting}[]
\DecValTok{1} \SpecialCharTok{+} \DecValTok{1}
\end{Highlighting}
\end{Shaded}

\begin{verbatim}
[1] 2
\end{verbatim}

The first three steps in selecting an appropriate sample size (created
2009-07-20, updated 2010-01-15) This page is moving to a new website.

I got an email last week from a client wanting to start a new research
project looking at relationships between parenting beliefs and childhood
behaviors. The description of the sorts of things to examine was quite
elaborate, and it ended with the question ``how many families would we
need to have any significant differences if they exist?'' Unfortunately,
all the elaborate information provided did not include the information I
would need to answer this question. Justifying a sample size usually
involves three steps.

\hypertarget{step-1-define-your-research-hypothesis.}{%
\section{Step 1: Define your research
hypothesis.}\label{step-1-define-your-research-hypothesis.}}

Not all research requires a research hypothesis, but most research does,
and until you define that hypothesis, it is impossible to make any
progress on calculating an appropriate sample size. This particular
email provided a fair amount of detail that could be used to derive a
hypothesis, but no formal hypothesis was directly stated. I like to use
the PICO format described in Evidence-Based Medicine to help people
formulate a good research hypothesis. A research hypothesis will usually
(but not always) have four elements:

P: patient population. This is the group of patients that you want to
examine.

I: intervention. This is what you do to the group of patients that you
think will help them improve

C: comparison group. This is the group of patients without the
intervention that you want to compare to.

O: outcome. This is the variable that will indicate whether or not the
intervention is successful.

Sometimes the intervention is not really something that you think will
help the patients but rather an exposure that the patients have to
endure that may produce some bad results.

A well-formulated hypothesis is important, because it tells the
statistician what type of statistic is likely to be needed to test the
hypothesis.

\hypertarget{what-do-you-do-if-you-dont-have-a-research-hypothesis}{%
\subsection{What do you do if you don't have a research
hypothesis?}\label{what-do-you-do-if-you-dont-have-a-research-hypothesis}}

In some research studies, the goal is exploratory. You don't have a
formal hypothesis at the start of the study, but rather you are hoping
that the data you collect will generate hypotheses for future studies.
The path to selecting a sample size in these settings is quite
different. Often you want to establish that the confidence intervals for
some of the key descriptive statistics in these studies has a reasonable
amount of precision.

\hypertarget{step-2-find-an-estimate-of-the-variability-of-your-outcome-measure.}{%
\section{Step 2: Find an estimate of the variability of your outcome
measure.}\label{step-2-find-an-estimate-of-the-variability-of-your-outcome-measure.}}

You've already done a literature review haven't you? If so, search
through the papers in your review that used the same outcome measure
that you are proposing in your study (the O in PICO). Ideally, the
outcome measure will be examined in a group of patients that is close to
the types of patients that you are studying (the P in PICO, or possibly
the C in PICO). This is not always easy, and you will sometimes be
forced to use a study where the patients are quite different from your
patients. Don't fret too much about this, but make a good faith effort
to find the most representative population that you can.

Some clients will raise an objection here and say that their research is
unique, so it is impossible to find a comparable paper. It is true that
most research is unique (otherwise it wouldn't be research). But what
these people are worried about is that their intervention (the I in
PICO) is unique. In these situations, the remainder of the hypothesis is
usually quite mundane: the patients, the comparison group, and the
outcome (P, C, and O in PICO) are all well studied. If you find a study
where the P, C, and O match reasonably well, but the I doesn't, then you
are probably going to get a good estimate of variation.

If there are major dissimilarities because this patient population (P)
is very different than any previously studied patient population, or
because the outcome measure (O) is newly developed by the researcher,
then perhaps a pilot study would be needed to establish a reasonable
estimate of variation.

Sometimes you can infer a standard deviation through general principles.
If a variable is constrained to be between 0 and 100, it would be
impossible, for example, for the standard deviation to be five thousand.
There are formulas relating the range of a distribution to the standard
deviation that can serve in a pinch if no other data is available. If
your outcome measure is a proportion, for example, then the variation is
related to the estimated proportion. Similarly, the variation in a count
variable is related to the mean of the counts. Find a paper that
establishes a proportion or average count in a control group similar to
your control group and any competent statistician will be able to get an
estimate of variation. In some situations, the amount of variation in a
proportion or count is larger than would be expected by the statistical
distributions (binomial and Poisson) traditionally associated with these
measures. Still, a calculation based on binomial or Poisson assumptions
is a reasonable starting point for further calculations.

\hypertarget{step-3-define-the-minimum-clinically-important-difference.}{%
\section{Step 3: define the minimum clinically important
difference.}\label{step-3-define-the-minimum-clinically-important-difference.}}

The minimum clinically significant difference is the boundary between a
difference so small that no one would adopt the new intervention on the
basis of such a meager changer and a difference large enough to make a
difference (that is, to convince people to change their behavior and
adopt the new therapy).

Establishing the minimum clinically relevant difference is a tricky
task, but it is something that should be done prior to any research
study. It's not easy but this is something that you have to do for
yourself. The clinically relevant difference is determined by medical
experts and not by statisticians. Hey, I'm still trying to understand
the difference between good and bad cholesterol; I wouldn't even be able
to start thinking about how much of a change in cholesterol is
considered clinically relevant. You might start by asking yourself ``How
much of an improvement would I have to see before I would adopt a new
treatment?'' Also, try talking with some of your colleagues. And look at
the size of improvements for other successful treatments.

For binary outcomes, the choice is not too difficult in theory. Suppose
that an intervention ``costs'' X dollars in the sense that it produces
that much pain, discomfort, and inconvenience, in addition to any direct
monetary costs. Suppose the value of a cure is kX where k is a number
greater than 1. A number less than 1, of course, means that even if you
could cure everyone, the costs outweigh the benefits of the cure.

For k\textgreater1, the minimum clinically significant difference in
proportions is 1/k. So if the cure is 10 times more valuable than the
costs, then you need to show at least a 10\% better cure rate (in
absolute terms) than no treatment or the current standard of treatment.
Otherwise, the cure is worse than the disease.

It helps to visualize this with certain types of alternative medicine.
If your treatment is aromatherapy, there is almost no cost involved, so
even a very slight probability of improvement might be worth it. But
Gerson therapy, which involves, among other things, coffee enemas, is a
different story. An enema is reasonably safe, but is not totally risk
free. And it involves a substantially greater level of inconvenience
than aromatherapy. So you'd only adopt Gerson therapy if it helped a
substantial fraction of patients. Exactly how many depends on the dollar
value that you place on having to endure a coffee enema, a task that I
will leave for someone else to quantify.

If there are side effects associated with the treatment that only occur
in a fraction of the patients receiving the treatment, then the
calculations are a bit trickier, but still possible in theory. It
becomes more tricky still when different people place different monetary
values on the risks and inconveniences of a new therapy.

For continuous variables, the minimum clinically significant difference
could be defined as above. Define a threshold that represents ``better''
versus ``not better'' and then try to shift the entire distribution so
that the fraction ``better'' under the new treatment is at least 1/k.

There have also been efforts to elucidate, through experiments,
interviews, and other approaches, what the average person considers an
important shift to be. For the visual analog scale of pain, for example,
a shift of at least 15 mm is considered the smallest value that is
noticeable to the average patient.

\bookmarksetup{startatroot}

\hypertarget{developing-your-pilot-study}{%
\chapter{Developing your pilot
study}\label{developing-your-pilot-study}}

Insert text.

\hypertarget{step-1.-visualize-the-bigger-study-in-the-future}{%
\section{Step 1. Visualize the bigger study in the
future}\label{step-1.-visualize-the-bigger-study-in-the-future}}

Insert text.

\hypertarget{step-2.-identify-planning-gaps}{%
\section{Step 2. Identify planning
gaps}\label{step-2.-identify-planning-gaps}}

Insert text.

\hypertarget{step-3.-think-about-where-murphys-law-might-strike}{%
\section{Step 3. Think about where ``Murphy's Law'' might
strike}\label{step-3.-think-about-where-murphys-law-might-strike}}

Insert text.

\hypertarget{the-fly-in-the-ointment-3}{%
\section{The fly in the ointment}\label{the-fly-in-the-ointment-3}}

Insert text.

\bookmarksetup{startatroot}

\hypertarget{designing-your-questionnaire}{%
\chapter{Designing your
questionnaire}\label{designing-your-questionnaire}}

Insert text.

\hypertarget{step-1.-brainstorm-a-long-list-and-then-prune-back}{%
\section{Step 1. Brainstorm a long list and then prune
back}\label{step-1.-brainstorm-a-long-list-and-then-prune-back}}

Insert text.

\hypertarget{step-2.-get-opinions-from-content-experts}{%
\section{Step 2. Get opinions from content
experts}\label{step-2.-get-opinions-from-content-experts}}

Insert text.

\hypertarget{step-3.-pilot-test-your-questionnaire}{%
\section{Step 3. Pilot test your
questionnaire}\label{step-3.-pilot-test-your-questionnaire}}

Insert text.

\hypertarget{the-fly-in-the-ointment-4}{%
\section{The fly in the ointment}\label{the-fly-in-the-ointment-4}}

Insert text.

\bookmarksetup{startatroot}

\hypertarget{obtaining-ethical-approval-for-your-study}{%
\chapter{Obtaining ethical approval for your
study}\label{obtaining-ethical-approval-for-your-study}}

Insert text.

\hypertarget{step-1.-establish-your-credentials-as-a-researcher}{%
\section{Step 1. Establish your credentials as a
researcher}\label{step-1.-establish-your-credentials-as-a-researcher}}

Insert text.

\hypertarget{step-2.-demonstrate-your-concern-for-privacy-rights}{%
\section{Step 2. Demonstrate your concern for privacy
rights}\label{step-2.-demonstrate-your-concern-for-privacy-rights}}

Insert text.

\hypertarget{step-3.-prepare-a-defensible-strategy-for-your-data-analysis}{%
\section{Step 3. Prepare a defensible strategy for your data
analysis}\label{step-3.-prepare-a-defensible-strategy-for-your-data-analysis}}

Insert text.

\hypertarget{the-fly-in-the-ointment-5}{%
\section{The fly in the ointment}\label{the-fly-in-the-ointment-5}}

Insert text.

\bookmarksetup{startatroot}

\hypertarget{setting-up-your-data-entry-process}{%
\chapter{Setting up your data entry
process}\label{setting-up-your-data-entry-process}}

This chapter is based on
\href{http://www.pmean.com/99/entry.html}{General guide to data entry}
published 1999-09-03. Also refer to
\href{http://www.pmean.com/10/Contents.html}{Tentative table of
contents}.

\hypertarget{step-1-arrange-your-data-in-a-rectangular-format}{%
\section{Step 1: Arrange your data in a rectangular
format}\label{step-1-arrange-your-data-in-a-rectangular-format}}

Add text.

\hypertarget{step-2-create-codes-for-categorical-data}{%
\section{Step 2: Create codes for categorical
data}\label{step-2-create-codes-for-categorical-data}}

Add text.

\hypertarget{step-3-document-missing-values}{%
\section{Step 3: Document missing
values}\label{step-3-document-missing-values}}

Add text.

\hypertarget{the-fly-in-the-ointment-longitudinal-and-repeated-measures-data}{%
\section{The fly in the ointment: longitudinal and repeated measures
data}\label{the-fly-in-the-ointment-longitudinal-and-repeated-measures-data}}

Add text.

\bookmarksetup{startatroot}

\hypertarget{setting-up-your-data-entry-process-1}{%
\chapter{Setting up your data entry
process}\label{setting-up-your-data-entry-process-1}}

This chapter is based on
\href{http://www.pmean.com/13/descriptive.html}{Steps in a descriptive
model} published 2001-10-11. Also refer to
\href{http://www.pmean.com/10/Contents.html}{Tentative table of
contents}.

\hypertarget{step-1-arrange-your-data-in-a-rectangular-format-1}{%
\section{Step 1: Arrange your data in a rectangular
format}\label{step-1-arrange-your-data-in-a-rectangular-format-1}}

Add text.

\hypertarget{step-2-create-codes-for-categorical-data-1}{%
\section{Step 2: Create codes for categorical
data}\label{step-2-create-codes-for-categorical-data-1}}

Add text.

\hypertarget{step-3-document-missing-values-1}{%
\section{Step 3: Document missing
values}\label{step-3-document-missing-values-1}}

Add text.

\hypertarget{the-fly-in-the-ointment-longitudinal-and-repeated-measures-data-1}{%
\section{The fly in the ointment: longitudinal and repeated measures
data}\label{the-fly-in-the-ointment-longitudinal-and-repeated-measures-data-1}}

Add text.

\bookmarksetup{startatroot}

\hypertarget{summary}{%
\chapter{Summary}\label{summary}}

In summary, this book has no content whatsoever.

\begin{Shaded}
\begin{Highlighting}[]
\DecValTok{1} \SpecialCharTok{+} \DecValTok{1}
\end{Highlighting}
\end{Shaded}

\begin{verbatim}
[1] 2
\end{verbatim}

\bookmarksetup{startatroot}

\hypertarget{references}{%
\chapter*{References}\label{references}}
\addcontentsline{toc}{chapter}{References}

\markboth{References}{References}

\hypertarget{refs}{}
\begin{CSLReferences}{1}{0}
\leavevmode\vadjust pre{\hypertarget{ref-knuth84}{}}%
Knuth, Donald E. 1984. {``Literate Programming.''} \emph{Comput. J.} 27
(2): 97--111. \url{https://doi.org/10.1093/comjnl/27.2.97}.

\end{CSLReferences}



\end{document}
