% Options for packages loaded elsewhere
\PassOptionsToPackage{unicode}{hyperref}
\PassOptionsToPackage{hyphens}{url}
\PassOptionsToPackage{dvipsnames,svgnames,x11names}{xcolor}
%
\documentclass[
  letterpaper,
  DIV=11,
  numbers=noendperiod]{scrreprt}

\usepackage{amsmath,amssymb}
\usepackage{iftex}
\ifPDFTeX
  \usepackage[T1]{fontenc}
  \usepackage[utf8]{inputenc}
  \usepackage{textcomp} % provide euro and other symbols
\else % if luatex or xetex
  \usepackage{unicode-math}
  \defaultfontfeatures{Scale=MatchLowercase}
  \defaultfontfeatures[\rmfamily]{Ligatures=TeX,Scale=1}
\fi
\usepackage{lmodern}
\ifPDFTeX\else  
    % xetex/luatex font selection
\fi
% Use upquote if available, for straight quotes in verbatim environments
\IfFileExists{upquote.sty}{\usepackage{upquote}}{}
\IfFileExists{microtype.sty}{% use microtype if available
  \usepackage[]{microtype}
  \UseMicrotypeSet[protrusion]{basicmath} % disable protrusion for tt fonts
}{}
\makeatletter
\@ifundefined{KOMAClassName}{% if non-KOMA class
  \IfFileExists{parskip.sty}{%
    \usepackage{parskip}
  }{% else
    \setlength{\parindent}{0pt}
    \setlength{\parskip}{6pt plus 2pt minus 1pt}}
}{% if KOMA class
  \KOMAoptions{parskip=half}}
\makeatother
\usepackage{xcolor}
\setlength{\emergencystretch}{3em} % prevent overfull lines
\setcounter{secnumdepth}{5}
% Make \paragraph and \subparagraph free-standing
\makeatletter
\ifx\paragraph\undefined\else
  \let\oldparagraph\paragraph
  \renewcommand{\paragraph}{
    \@ifstar
      \xxxParagraphStar
      \xxxParagraphNoStar
  }
  \newcommand{\xxxParagraphStar}[1]{\oldparagraph*{#1}\mbox{}}
  \newcommand{\xxxParagraphNoStar}[1]{\oldparagraph{#1}\mbox{}}
\fi
\ifx\subparagraph\undefined\else
  \let\oldsubparagraph\subparagraph
  \renewcommand{\subparagraph}{
    \@ifstar
      \xxxSubParagraphStar
      \xxxSubParagraphNoStar
  }
  \newcommand{\xxxSubParagraphStar}[1]{\oldsubparagraph*{#1}\mbox{}}
  \newcommand{\xxxSubParagraphNoStar}[1]{\oldsubparagraph{#1}\mbox{}}
\fi
\makeatother

\usepackage{color}
\usepackage{fancyvrb}
\newcommand{\VerbBar}{|}
\newcommand{\VERB}{\Verb[commandchars=\\\{\}]}
\DefineVerbatimEnvironment{Highlighting}{Verbatim}{commandchars=\\\{\}}
% Add ',fontsize=\small' for more characters per line
\usepackage{framed}
\definecolor{shadecolor}{RGB}{241,243,245}
\newenvironment{Shaded}{\begin{snugshade}}{\end{snugshade}}
\newcommand{\AlertTok}[1]{\textcolor[rgb]{0.68,0.00,0.00}{#1}}
\newcommand{\AnnotationTok}[1]{\textcolor[rgb]{0.37,0.37,0.37}{#1}}
\newcommand{\AttributeTok}[1]{\textcolor[rgb]{0.40,0.45,0.13}{#1}}
\newcommand{\BaseNTok}[1]{\textcolor[rgb]{0.68,0.00,0.00}{#1}}
\newcommand{\BuiltInTok}[1]{\textcolor[rgb]{0.00,0.23,0.31}{#1}}
\newcommand{\CharTok}[1]{\textcolor[rgb]{0.13,0.47,0.30}{#1}}
\newcommand{\CommentTok}[1]{\textcolor[rgb]{0.37,0.37,0.37}{#1}}
\newcommand{\CommentVarTok}[1]{\textcolor[rgb]{0.37,0.37,0.37}{\textit{#1}}}
\newcommand{\ConstantTok}[1]{\textcolor[rgb]{0.56,0.35,0.01}{#1}}
\newcommand{\ControlFlowTok}[1]{\textcolor[rgb]{0.00,0.23,0.31}{\textbf{#1}}}
\newcommand{\DataTypeTok}[1]{\textcolor[rgb]{0.68,0.00,0.00}{#1}}
\newcommand{\DecValTok}[1]{\textcolor[rgb]{0.68,0.00,0.00}{#1}}
\newcommand{\DocumentationTok}[1]{\textcolor[rgb]{0.37,0.37,0.37}{\textit{#1}}}
\newcommand{\ErrorTok}[1]{\textcolor[rgb]{0.68,0.00,0.00}{#1}}
\newcommand{\ExtensionTok}[1]{\textcolor[rgb]{0.00,0.23,0.31}{#1}}
\newcommand{\FloatTok}[1]{\textcolor[rgb]{0.68,0.00,0.00}{#1}}
\newcommand{\FunctionTok}[1]{\textcolor[rgb]{0.28,0.35,0.67}{#1}}
\newcommand{\ImportTok}[1]{\textcolor[rgb]{0.00,0.46,0.62}{#1}}
\newcommand{\InformationTok}[1]{\textcolor[rgb]{0.37,0.37,0.37}{#1}}
\newcommand{\KeywordTok}[1]{\textcolor[rgb]{0.00,0.23,0.31}{\textbf{#1}}}
\newcommand{\NormalTok}[1]{\textcolor[rgb]{0.00,0.23,0.31}{#1}}
\newcommand{\OperatorTok}[1]{\textcolor[rgb]{0.37,0.37,0.37}{#1}}
\newcommand{\OtherTok}[1]{\textcolor[rgb]{0.00,0.23,0.31}{#1}}
\newcommand{\PreprocessorTok}[1]{\textcolor[rgb]{0.68,0.00,0.00}{#1}}
\newcommand{\RegionMarkerTok}[1]{\textcolor[rgb]{0.00,0.23,0.31}{#1}}
\newcommand{\SpecialCharTok}[1]{\textcolor[rgb]{0.37,0.37,0.37}{#1}}
\newcommand{\SpecialStringTok}[1]{\textcolor[rgb]{0.13,0.47,0.30}{#1}}
\newcommand{\StringTok}[1]{\textcolor[rgb]{0.13,0.47,0.30}{#1}}
\newcommand{\VariableTok}[1]{\textcolor[rgb]{0.07,0.07,0.07}{#1}}
\newcommand{\VerbatimStringTok}[1]{\textcolor[rgb]{0.13,0.47,0.30}{#1}}
\newcommand{\WarningTok}[1]{\textcolor[rgb]{0.37,0.37,0.37}{\textit{#1}}}

\providecommand{\tightlist}{%
  \setlength{\itemsep}{0pt}\setlength{\parskip}{0pt}}\usepackage{longtable,booktabs,array}
\usepackage{calc} % for calculating minipage widths
% Correct order of tables after \paragraph or \subparagraph
\usepackage{etoolbox}
\makeatletter
\patchcmd\longtable{\par}{\if@noskipsec\mbox{}\fi\par}{}{}
\makeatother
% Allow footnotes in longtable head/foot
\IfFileExists{footnotehyper.sty}{\usepackage{footnotehyper}}{\usepackage{footnote}}
\makesavenoteenv{longtable}
\usepackage{graphicx}
\makeatletter
\def\maxwidth{\ifdim\Gin@nat@width>\linewidth\linewidth\else\Gin@nat@width\fi}
\def\maxheight{\ifdim\Gin@nat@height>\textheight\textheight\else\Gin@nat@height\fi}
\makeatother
% Scale images if necessary, so that they will not overflow the page
% margins by default, and it is still possible to overwrite the defaults
% using explicit options in \includegraphics[width, height, ...]{}
\setkeys{Gin}{width=\maxwidth,height=\maxheight,keepaspectratio}
% Set default figure placement to htbp
\makeatletter
\def\fps@figure{htbp}
\makeatother
% definitions for citeproc citations
\NewDocumentCommand\citeproctext{}{}
\NewDocumentCommand\citeproc{mm}{%
  \begingroup\def\citeproctext{#2}\cite{#1}\endgroup}
\makeatletter
 % allow citations to break across lines
 \let\@cite@ofmt\@firstofone
 % avoid brackets around text for \cite:
 \def\@biblabel#1{}
 \def\@cite#1#2{{#1\if@tempswa , #2\fi}}
\makeatother
\newlength{\cslhangindent}
\setlength{\cslhangindent}{1.5em}
\newlength{\csllabelwidth}
\setlength{\csllabelwidth}{3em}
\newenvironment{CSLReferences}[2] % #1 hanging-indent, #2 entry-spacing
 {\begin{list}{}{%
  \setlength{\itemindent}{0pt}
  \setlength{\leftmargin}{0pt}
  \setlength{\parsep}{0pt}
  % turn on hanging indent if param 1 is 1
  \ifodd #1
   \setlength{\leftmargin}{\cslhangindent}
   \setlength{\itemindent}{-1\cslhangindent}
  \fi
  % set entry spacing
  \setlength{\itemsep}{#2\baselineskip}}}
 {\end{list}}
\usepackage{calc}
\newcommand{\CSLBlock}[1]{\hfill\break\parbox[t]{\linewidth}{\strut\ignorespaces#1\strut}}
\newcommand{\CSLLeftMargin}[1]{\parbox[t]{\csllabelwidth}{\strut#1\strut}}
\newcommand{\CSLRightInline}[1]{\parbox[t]{\linewidth - \csllabelwidth}{\strut#1\strut}}
\newcommand{\CSLIndent}[1]{\hspace{\cslhangindent}#1}

\KOMAoption{captions}{tableheading}
\makeatletter
\@ifpackageloaded{bookmark}{}{\usepackage{bookmark}}
\makeatother
\makeatletter
\@ifpackageloaded{caption}{}{\usepackage{caption}}
\AtBeginDocument{%
\ifdefined\contentsname
  \renewcommand*\contentsname{Table of contents}
\else
  \newcommand\contentsname{Table of contents}
\fi
\ifdefined\listfigurename
  \renewcommand*\listfigurename{List of Figures}
\else
  \newcommand\listfigurename{List of Figures}
\fi
\ifdefined\listtablename
  \renewcommand*\listtablename{List of Tables}
\else
  \newcommand\listtablename{List of Tables}
\fi
\ifdefined\figurename
  \renewcommand*\figurename{Figure}
\else
  \newcommand\figurename{Figure}
\fi
\ifdefined\tablename
  \renewcommand*\tablename{Table}
\else
  \newcommand\tablename{Table}
\fi
}
\@ifpackageloaded{float}{}{\usepackage{float}}
\floatstyle{ruled}
\@ifundefined{c@chapter}{\newfloat{codelisting}{h}{lop}}{\newfloat{codelisting}{h}{lop}[chapter]}
\floatname{codelisting}{Listing}
\newcommand*\listoflistings{\listof{codelisting}{List of Listings}}
\makeatother
\makeatletter
\makeatother
\makeatletter
\@ifpackageloaded{caption}{}{\usepackage{caption}}
\@ifpackageloaded{subcaption}{}{\usepackage{subcaption}}
\makeatother

\ifLuaTeX
  \usepackage{selnolig}  % disable illegal ligatures
\fi
\usepackage{bookmark}

\IfFileExists{xurl.sty}{\usepackage{xurl}}{} % add URL line breaks if available
\urlstyle{same} % disable monospaced font for URLs
\hypersetup{
  pdftitle={jump-start-book},
  pdfauthor={Steve Simon},
  colorlinks=true,
  linkcolor={blue},
  filecolor={Maroon},
  citecolor={Blue},
  urlcolor={Blue},
  pdfcreator={LaTeX via pandoc}}


\title{jump-start-book}
\author{Steve Simon}
\date{2023-10-03}

\begin{document}
\maketitle

\renewcommand*\contentsname{Table of contents}
{
\hypersetup{linkcolor=}
\setcounter{tocdepth}{2}
\tableofcontents
}

\bookmarksetup{startatroot}

\chapter*{Preface}\label{preface}
\addcontentsline{toc}{chapter}{Preface}

\markboth{Preface}{Preface}

This book was a long time coming.

\bookmarksetup{startatroot}

\chapter{Introduction}\label{introduction}

This is a great book. Some of the material here is reproduced from my
\href{http://www.pmean.com/10/SecondBook.html}{original book proposal},
published in 2010-07-23.

Refer to the \href{http://www.pmean.com/10/Contents.html}{tentative
table of contents}, published 2010-07-24.

This book will explain the next three steps that you have to take at any
stage of your research project and it is targeted to beginning
researchers who are often confused about the research process and are
often unsure about what to do next.

When people come into my office asking for advice about Statistics, they
may be at the beginning, the planning phase of the study. Or they may be
getting their data ready for data analysis. Or they may be figuring out
which data analysis they are supposed to be using. Or they may be
thinking about how to write up all the results. The one thing in common
is that they come to see me when they are ``stuck.'' There are a few
exceptions, people who know what they want to do next and they just want
to run their thoughts by me to get my opinion. But most people, if they
knew what they were supposed to be doing, they'd be doing those things.

So what do I advise people to do when they are stuck? I can't lay out
the entire task in front of them, but I can almost always tell them what
the next few steps should be. If they take their first three steps
carefully, the following steps should eventually become obvious.

I am writing this book to reach people who can't visit me in person.
This book is for you if you have to struggle with a research project,
especially your first major research project, and you want guidance on
how to best proceed. The examples I give will be targeted to a medical
research setting and to studies involving humans, but should be broadly
applicable to other research areas, especially research in the social
sciences.

There's a certain amount of arrogance in writing a book and expecting
people to pay good money to read it. I'm certainly arrogant enough, but
I hope this book will justify that arrogance. I do have more than 25
years of consulting experience in academic settings, in the federal
government, in a hospital, and as an independent consultant. The one
advantage of being old is that I've seen it all before. I could not have
written this book twenty or even ten years ago.

There's a second level of arrogance, though, in presupposing that I can
offer advice to you without ever having met you and without knowing
anything about your research project. There are certainly some research
projects where the steps I suggest may not make sense, and I apologize
in advance to anyone who has such a project and does not get any benefit
from this book. I do believe, however, that there is a commonality in
most research. In particular, while the final steps in a data analysis
might be impossible to predict, the initial steps can be largely
predicted, based on my experience. And that's what most people need.
Once they get some momentum back in their research project, they usually
find a way to finish things up.

The general advice behind the steps I am suggesting is that you should
never dive into the deep end of the swimming pool first. There's a
natural tendency to tackle the hardest thing first, but this is a
mistake. Instead, wade in gently at the shallow end. Thus, the first
step for a descriptive data analysis is ``know your count'' which means
to know how many observations are in your data set and how many values
are missing in each of your key variables. This sounds like a trivial
step, but you must do this. If you don't know how much data you have,
then you will be likely to overlook important details later on like how
complex of a statistical model might be supported by your data set.

This is not just advice that I offer to you, but advice I follow myself.
When I help out with a new research problem or a new data set, I can't
jump in the deep end either. I need to get comfortable with things.
Small easy steps will help build my confidence before I tackle the big
things.

This book covers the full range of research. The first few chapters talk
about the steps you need to take in designing your research study. A
good research plan, put in writing, is essential for quality research.
No one asks me about how to collect the data, but once they have it,
they need to know how to enter it into the computer, if it is on paper,
or how to import the data into a software package if it comes from an
electronic source.

Once you have data in a program like SPSS, you may not know what
procedures to use. I have chapters on how to start up a descriptive data
analysis (the foundation of all other data analyses), and how to begin
more complex data analyses like linear regression, logistic regression,
and survival data analysis. I use SPSS in my examples for these chapters
because I think it is an ideal statistical package for beginners,
because it is widely available in many academic and medical centers, and
it is easy to explain. But the general principles apply if you are using
SAS, STATA, or any other statistical package.

Finally, once you have all your data analysis, you need to start writing
up your results. I have chapters on how to write the methods section,
results section, and discussion section of a typical research paper.

I can't talk about purely scientific issues. If you're stuck because you
can't get your flow cytometer to work, you won't find any help here.
There is, of course, substantial overlap between science and statistics,
so I won't shy away from talking about this entirely. Just keep in mind
that my comments regarding science are as an outsider and that I do not
have any special expertise except for what little that has rubbed off on
me from the very intelligent scientists and doctors that I have
collaborated with.

There are book out there that offer a more comprehensive overview of
each of these steps. If you are writing a questionnaire, for example,
Alreck and Settle offer complete and thorough advice on what to do.
There are a wide range of books about how to run SPSS, and Julie Pallant
has an excellent one for beginners. If you are interested in what
statistical tests to use when, then Norman and Streiner have an
entertaining and informative book. If you are writing up results for
publication, then look no further than Lang and Secic, a definitive
guide with examples from the published literature and comprehensive
checklists of all the things you need.

What's different about my book is that I am not trying to be a
comprehensive guide that explains every single step that you might
possibly take. Those guides are important but they are also difficult to
read when your concern is not with every single step of the process but
rather in deciding what you should be doing right now.

Each chapter starts with an introduction and then defines the first
three steps that you should take at this stage. Finally, I list some
special cases that can cause special difficulties (the fly in the
ointment) and which might require a more intense interaction with your
statistical consultant.

Fear is often a paralyzing emotion, and it may be just fear itself that
is causing your difficulties. Research is not easy, but don't fool
yourself into thinking that it is beyond your capabilities. You have to
be a very smart person to be in a position where you are capable of
doing an independent research project, so you are more capable than you
might believe. You can get your research project started, and if it is
stalled, you can get a jumpstart that will help you get moving again.
Just focus on what your next three steps should be.

\bookmarksetup{startatroot}

\chapter{Picking a research topic}\label{picking-a-research-topic}

Insert text.

\section{Step 1. Describe what bothers
you}\label{step-1.-describe-what-bothers-you}

Insert text.

\section{Step 2. Read what others have
said}\label{step-2.-read-what-others-have-said}

Insert text.

\section{Step 3. Talk about your idea with
others}\label{step-3.-talk-about-your-idea-with-others}

Insert text.

\section{The fly in the ointment}\label{the-fly-in-the-ointment}

Insert text.

\bookmarksetup{startatroot}

\chapter{Selecting your research
hypothesis}\label{selecting-your-research-hypothesis}

Insert text.

\section{Step 1. Choose an outcome
measure}\label{step-1.-choose-an-outcome-measure}

Insert text.

\section{Step 2. Define your patient
population}\label{step-2.-define-your-patient-population}

Insert text.

\section{Step 3. Select your control
group}\label{step-3.-select-your-control-group}

Insert text.

\section{The fly in the ointment}\label{the-fly-in-the-ointment-1}

Insert text.

\bookmarksetup{startatroot}

\chapter{Selecting your research
design}\label{selecting-your-research-design}

Insert text.

\section{Step 1. Decide if you can/should
randomize}\label{step-1.-decide-if-you-canshould-randomize}

Insert text.

\section{Step 2. Consider the advantages/disadvantages of retrospective
versus prospective
research}\label{step-2.-consider-the-advantagesdisadvantages-of-retrospective-versus-prospective-research}

Insert text.

\section{Step 3. Think about the most efficient way to select
subjects}\label{step-3.-think-about-the-most-efficient-way-to-select-subjects}

Insert text.

\section{The fly in the ointment}\label{the-fly-in-the-ointment-2}

Insert text.

\bookmarksetup{startatroot}

\chapter{Selecting your sample size}\label{selecting-your-sample-size}

This is a book created from markdown and executable code.

See Knuth (1984) for additional discussion of literate programming.

\begin{Shaded}
\begin{Highlighting}[]
\DecValTok{1} \SpecialCharTok{+} \DecValTok{1}
\end{Highlighting}
\end{Shaded}

\begin{verbatim}
[1] 2
\end{verbatim}

The first three steps in selecting an appropriate sample size (created
2009-07-20, updated 2010-01-15) This page is moving to a new website.

I got an email last week from a client wanting to start a new research
project looking at relationships between parenting beliefs and childhood
behaviors. The description of the sorts of things to examine was quite
elaborate, and it ended with the question ``how many families would we
need to have any significant differences if they exist?'' Unfortunately,
all the elaborate information provided did not include the information I
would need to answer this question. Justifying a sample size usually
involves three steps.

\section{Step 1: Define your research
hypothesis.}\label{step-1-define-your-research-hypothesis.}

Not all research requires a research hypothesis, but most research does,
and until you define that hypothesis, it is impossible to make any
progress on calculating an appropriate sample size. This particular
email provided a fair amount of detail that could be used to derive a
hypothesis, but no formal hypothesis was directly stated. I like to use
the PICO format described in Evidence-Based Medicine to help people
formulate a good research hypothesis. A research hypothesis will usually
(but not always) have four elements:

P: patient population. This is the group of patients that you want to
examine.

I: intervention. This is what you do to the group of patients that you
think will help them improve

C: comparison group. This is the group of patients without the
intervention that you want to compare to.

O: outcome. This is the variable that will indicate whether or not the
intervention is successful.

Sometimes the intervention is not really something that you think will
help the patients but rather an exposure that the patients have to
endure that may produce some bad results.

A well-formulated hypothesis is important, because it tells the
statistician what type of statistic is likely to be needed to test the
hypothesis.

\subsection{What do you do if you don't have a research
hypothesis?}\label{what-do-you-do-if-you-dont-have-a-research-hypothesis}

In some research studies, the goal is exploratory. You don't have a
formal hypothesis at the start of the study, but rather you are hoping
that the data you collect will generate hypotheses for future studies.
The path to selecting a sample size in these settings is quite
different. Often you want to establish that the confidence intervals for
some of the key descriptive statistics in these studies has a reasonable
amount of precision.

\section{Step 2: Find an estimate of the variability of your outcome
measure.}\label{step-2-find-an-estimate-of-the-variability-of-your-outcome-measure.}

You've already done a literature review haven't you? If so, search
through the papers in your review that used the same outcome measure
that you are proposing in your study (the O in PICO). Ideally, the
outcome measure will be examined in a group of patients that is close to
the types of patients that you are studying (the P in PICO, or possibly
the C in PICO). This is not always easy, and you will sometimes be
forced to use a study where the patients are quite different from your
patients. Don't fret too much about this, but make a good faith effort
to find the most representative population that you can.

Some clients will raise an objection here and say that their research is
unique, so it is impossible to find a comparable paper. It is true that
most research is unique (otherwise it wouldn't be research). But what
these people are worried about is that their intervention (the I in
PICO) is unique. In these situations, the remainder of the hypothesis is
usually quite mundane: the patients, the comparison group, and the
outcome (P, C, and O in PICO) are all well studied. If you find a study
where the P, C, and O match reasonably well, but the I doesn't, then you
are probably going to get a good estimate of variation.

If there are major dissimilarities because this patient population (P)
is very different than any previously studied patient population, or
because the outcome measure (O) is newly developed by the researcher,
then perhaps a pilot study would be needed to establish a reasonable
estimate of variation.

Sometimes you can infer a standard deviation through general principles.
If a variable is constrained to be between 0 and 100, it would be
impossible, for example, for the standard deviation to be five thousand.
There are formulas relating the range of a distribution to the standard
deviation that can serve in a pinch if no other data is available. If
your outcome measure is a proportion, for example, then the variation is
related to the estimated proportion. Similarly, the variation in a count
variable is related to the mean of the counts. Find a paper that
establishes a proportion or average count in a control group similar to
your control group and any competent statistician will be able to get an
estimate of variation. In some situations, the amount of variation in a
proportion or count is larger than would be expected by the statistical
distributions (binomial and Poisson) traditionally associated with these
measures. Still, a calculation based on binomial or Poisson assumptions
is a reasonable starting point for further calculations.

\section{Step 3: define the minimum clinically important
difference.}\label{step-3-define-the-minimum-clinically-important-difference.}

The minimum clinically significant difference is the boundary between a
difference so small that no one would adopt the new intervention on the
basis of such a meager changer and a difference large enough to make a
difference (that is, to convince people to change their behavior and
adopt the new therapy).

Establishing the minimum clinically relevant difference is a tricky
task, but it is something that should be done prior to any research
study. It's not easy but this is something that you have to do for
yourself. The clinically relevant difference is determined by medical
experts and not by statisticians. Hey, I'm still trying to understand
the difference between good and bad cholesterol; I wouldn't even be able
to start thinking about how much of a change in cholesterol is
considered clinically relevant. You might start by asking yourself ``How
much of an improvement would I have to see before I would adopt a new
treatment?'' Also, try talking with some of your colleagues. And look at
the size of improvements for other successful treatments.

For binary outcomes, the choice is not too difficult in theory. Suppose
that an intervention ``costs'' X dollars in the sense that it produces
that much pain, discomfort, and inconvenience, in addition to any direct
monetary costs. Suppose the value of a cure is kX where k is a number
greater than 1. A number less than 1, of course, means that even if you
could cure everyone, the costs outweigh the benefits of the cure.

For k\textgreater1, the minimum clinically significant difference in
proportions is 1/k. So if the cure is 10 times more valuable than the
costs, then you need to show at least a 10\% better cure rate (in
absolute terms) than no treatment or the current standard of treatment.
Otherwise, the cure is worse than the disease.

It helps to visualize this with certain types of alternative medicine.
If your treatment is aromatherapy, there is almost no cost involved, so
even a very slight probability of improvement might be worth it. But
Gerson therapy, which involves, among other things, coffee enemas, is a
different story. An enema is reasonably safe, but is not totally risk
free. And it involves a substantially greater level of inconvenience
than aromatherapy. So you'd only adopt Gerson therapy if it helped a
substantial fraction of patients. Exactly how many depends on the dollar
value that you place on having to endure a coffee enema, a task that I
will leave for someone else to quantify.

If there are side effects associated with the treatment that only occur
in a fraction of the patients receiving the treatment, then the
calculations are a bit trickier, but still possible in theory. It
becomes more tricky still when different people place different monetary
values on the risks and inconveniences of a new therapy.

For continuous variables, the minimum clinically significant difference
could be defined as above. Define a threshold that represents ``better''
versus ``not better'' and then try to shift the entire distribution so
that the fraction ``better'' under the new treatment is at least 1/k.

There have also been efforts to elucidate, through experiments,
interviews, and other approaches, what the average person considers an
important shift to be. For the visual analog scale of pain, for example,
a shift of at least 15 mm is considered the smallest value that is
noticeable to the average patient.

\bookmarksetup{startatroot}

\chapter{Developing your pilot study}\label{developing-your-pilot-study}

Insert text.

\section{Step 1. Visualize the bigger study in the
future}\label{step-1.-visualize-the-bigger-study-in-the-future}

Insert text.

\section{Step 2. Identify planning
gaps}\label{step-2.-identify-planning-gaps}

Insert text.

\section{Step 3. Think about where ``Murphy's Law'' might
strike}\label{step-3.-think-about-where-murphys-law-might-strike}

Insert text.

\section{The fly in the ointment}\label{the-fly-in-the-ointment-3}

Insert text.

\bookmarksetup{startatroot}

\chapter{Designing your
questionnaire}\label{designing-your-questionnaire}

Insert text.

\section{Step 1. Brainstorm a long list and then prune
back}\label{step-1.-brainstorm-a-long-list-and-then-prune-back}

Insert text.

\section{Step 2. Get opinions from content
experts}\label{step-2.-get-opinions-from-content-experts}

Insert text.

\section{Step 3. Pilot test your
questionnaire}\label{step-3.-pilot-test-your-questionnaire}

Insert text.

\section{The fly in the ointment}\label{the-fly-in-the-ointment-4}

Insert text.

\bookmarksetup{startatroot}

\chapter{Obtaining ethical approval for your
study}\label{obtaining-ethical-approval-for-your-study}

Insert text.

\section{Step 1. Establish your credentials as a
researcher}\label{step-1.-establish-your-credentials-as-a-researcher}

Insert text.

\section{Step 2. Demonstrate your concern for privacy
rights}\label{step-2.-demonstrate-your-concern-for-privacy-rights}

Insert text.

\section{Step 3. Prepare a defensible strategy for your data
analysis}\label{step-3.-prepare-a-defensible-strategy-for-your-data-analysis}

Insert text.

\section{The fly in the ointment}\label{the-fly-in-the-ointment-5}

Insert text.

\bookmarksetup{startatroot}

\chapter{Setting up your data entry
process}\label{setting-up-your-data-entry-process}

This chapter is based on
\href{http://www.pmean.com/99/entry.html}{General guide to data entry}
published 1999-09-03.

\section{Step 1: Arrange your data in a rectangular
format}\label{step-1-arrange-your-data-in-a-rectangular-format}

Add text.

\section{Step 2: Create codes for categorical
data}\label{step-2-create-codes-for-categorical-data}

Add text.

\section{Step 3: Document missing
values}\label{step-3-document-missing-values}

Add text.

\section{The fly in the ointment: longitudinal and repeated measures
data}\label{the-fly-in-the-ointment-longitudinal-and-repeated-measures-data}

Add text.

\bookmarksetup{startatroot}

\chapter{Running a descriptive data
analysis}\label{running-a-descriptive-data-analysis}

This chapter is based on
\href{http://new.pmean.com/steps-in-descriptive-model/}{Steps in a
descriptive model} published 2001-10-11.

\section{Step 1: Know your count}\label{step-1-know-your-count}

Add text.

\section{Step 2: Compute ranges and
frequencies}\label{step-2-compute-ranges-and-frequencies}

Add text.

\section{Step 3: Examine relationships using crosstabs, boxplots, and/or
scatterplots}\label{step-3-examine-relationships-using-crosstabs-boxplots-andor-scatterplots}

Add text.

\section{The fly in the ointment: ordinal
data}\label{the-fly-in-the-ointment-ordinal-data}

Add text.

\bookmarksetup{startatroot}

\chapter{Running a linear regression
analysis}\label{running-a-linear-regression-analysis}

This chapter is based on
\href{http://new.pmean.com/steps-in-linear-regression/}{Steps in a
linear regression analysis} published 2001-10-11.

\section{Step 1: Plot your data}\label{step-1-plot-your-data}

Plot your data and examine your assumptions.

Add text.

\section{Step 2: Compute a simple
model}\label{step-2-compute-a-simple-model}

Add text.

\section{Step 3: Compute an adjusted
model}\label{step-3-compute-an-adjusted-model}

Add text.

\section{The fly in the ointment: clustered
data}\label{the-fly-in-the-ointment-clustered-data}

Add text.

\bookmarksetup{startatroot}

\chapter{Running a logistic regression
analysis}\label{running-a-logistic-regression-analysis}

This chapter is based on
\href{http://new.pmean.com/steps-in-linear-regression/}{Steps in a
linear regression analysis} published 2001-10-11.

\section{Step 1: Examine raw
probabilities}\label{step-1-examine-raw-probabilities}

Examine raw probabilities using crosstabs

Add text.

\section{Step 2: Compute a simple
model}\label{step-2-compute-a-simple-model-1}

Add text.

\section{Step 3: Compute an adjusted
model}\label{step-3-compute-an-adjusted-model-1}

Add text.

\section{The fly in the ointment: clustered
data}\label{the-fly-in-the-ointment-clustered-data-1}

Add text.

\bookmarksetup{startatroot}

\chapter{Running a multi-factor analysis of variance
model}\label{running-a-multi-factor-analysis-of-variance-model}

This chapter is based on \href{http://www.pmean.com/posts/anova/}{Steps
in a typical ANOVA model} published 2002-10-11.

\section{Step 1: Plot clustered
boxplots}\label{step-1-plot-clustered-boxplots}

Look for interactions

\section{Step 2: Fit individual
models}\label{step-2-fit-individual-models}

Add text.

\section{Step 3: Fit multi-factor
models}\label{step-3-fit-multi-factor-models}

Add text.

\section{The fly in the ointment: unbalanced
data}\label{the-fly-in-the-ointment-unbalanced-data}

Add text.

\bookmarksetup{startatroot}

\chapter{Running a survival data
analysis}\label{running-a-survival-data-analysis}

This chapter is based on
\href{http://www.new.pmean.com/steps-in-survival-analysis/}{Steps in a
typical survival data analysis} published 2002-10-11.

\section{Step 1: Plot Kaplan-Meier curves for each
group}\label{step-1-plot-kaplan-meier-curves-for-each-group}

Examine raw probabilities using crosstabs

Add text.

\section{Step 2: Compute a simple surivival
model}\label{step-2-compute-a-simple-surivival-model}

Add text.

\section{Step 3: Compute an adjusted survival
model}\label{step-3-compute-an-adjusted-survival-model}

Add text.

\section{The fly in the ointment: time-varying
covariates}\label{the-fly-in-the-ointment-time-varying-covariates}

Add text.

\bookmarksetup{startatroot}

\chapter{Running a non-linear regression
analysis}\label{running-a-non-linear-regression-analysis}

This chapter is based on
\href{http://new.pmean.com/fitting-s-shaped-curves/}{S-shaped curves}
published 2004-02-12.

\section{Step 1: Plot your data}\label{step-1-plot-your-data-1}

Look for interactions

\section{Step 2: Explore your
function}\label{step-2-explore-your-function}

Add text.

\section{Step 3: Fit your model}\label{step-3-fit-your-model}

Add text.

\section{The fly in the ointment: ill-conditioned
models}\label{the-fly-in-the-ointment-ill-conditioned-models}

Add text.

\bookmarksetup{startatroot}

\chapter{Running a systematic
overview}\label{running-a-systematic-overview}

This chapter is based on {[}So you're thinking about a systematic
overview{]}{[}sim\_sys{]} published 2016-06-17.

\section{Step 1: Write a protocol}\label{step-1-write-a-protocol}

Look for interactions

\section{Step 2: Collect references}\label{step-2-collect-references}

Add text.

\section{Step 3: Extract information}\label{step-3-extract-information}

Add text.

\section{The fly in the ointment: extreme
heterogeneity}\label{the-fly-in-the-ointment-extreme-heterogeneity}

Add text.

{[}sim-sys{]}

\bookmarksetup{startatroot}

\chapter{Running a focus group}\label{running-a-focus-group}

This chapter is based on {[}Focus groups and qualitative
research{]}{[}sim\_foc{]} published 2004-04-13.

\section{Step 1: Select a theoretical
framework}\label{step-1-select-a-theoretical-framework}

Look for interactions

\section{Step 2: Write probe
questions}\label{step-2-write-probe-questions}

Add text.

\section{Step 3: Revise your questions as you learn from each focus
group}\label{step-3-revise-your-questions-as-you-learn-from-each-focus-group}

Add text.

\section{The fly in the ointment:uneven
participation}\label{the-fly-in-the-ointmentuneven-participation}

Add text.

\bookmarksetup{startatroot}

\chapter{Writing a literature review}\label{writing-a-literature-review}

There are two relevant pages:

\begin{itemize}
\tightlist
\item
  \href{http://new.pmean.com/writing-methods-section/}{Writing the
  introduction section} published 2019-03-29.
\item
  {[}How to write a literature review{]}{[}sim\_lit{]} published
  2019-01-16
\end{itemize}

\section{Step 1: Organize your
references}\label{step-1-organize-your-references}

Look for interactions

\section{Step 2: Decide on an organizing
principle}\label{step-2-decide-on-an-organizing-principle}

Add text.

\section{Step 3: Build transitions}\label{step-3-build-transitions}

Add text.

\section{The fly in the ointment: too many/too few
references}\label{the-fly-in-the-ointment-too-manytoo-few-references}

Add text.

\bookmarksetup{startatroot}

\chapter{Writing a methods section}\label{writing-a-methods-section}

This is based on {[}Writing the methods section of a research
paper{]}{[}sim\_met{]} published 2019-04-25.

\section{Step 1:}\label{step-1}

Add text.

\section{Step 2:}\label{step-2}

Add text.

\section{Step 3:}\label{step-3}

Add text.

\section{The fly in the ointment:}\label{the-fly-in-the-ointment-6}

Add text.

\bookmarksetup{startatroot}

\chapter{Writing a results section}\label{writing-a-results-section}

I have not found any pages on my website yet for this topic.

\section{Step 1:}\label{step-1-1}

Add text.

\section{Step 2:}\label{step-2-1}

Add text.

\section{Step 3:}\label{step-3-1}

Add text.

\section{The fly in the ointment:}\label{the-fly-in-the-ointment-7}

Add text.

\bookmarksetup{startatroot}

\chapter{Writing a discussion
section}\label{writing-a-discussion-section}

I have not found any pages on my website yet for this topic.

\section{Step 1:}\label{step-1-2}

Add text.

\section{Step 2:}\label{step-2-2}

Add text.

\section{Step 3:}\label{step-3-2}

Add text.

\section{The fly in the ointment:}\label{the-fly-in-the-ointment-8}

Add text.

\bookmarksetup{startatroot}

\chapter{Summary}\label{summary}

In summary, this book has no content whatsoever.

\begin{Shaded}
\begin{Highlighting}[]
\DecValTok{1} \SpecialCharTok{+} \DecValTok{1}
\end{Highlighting}
\end{Shaded}

\begin{verbatim}
[1] 2
\end{verbatim}

\bookmarksetup{startatroot}

\chapter*{References}\label{references}
\addcontentsline{toc}{chapter}{References}

\markboth{References}{References}

\phantomsection\label{refs}
\begin{CSLReferences}{1}{0}
\bibitem[\citeproctext]{ref-knuth84}
Knuth, Donald E. 1984. {``Literate Programming.''} \emph{Comput. J.} 27
(2): 97--111. \url{https://doi.org/10.1093/comjnl/27.2.97}.

\end{CSLReferences}




\end{document}
